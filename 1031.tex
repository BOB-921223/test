\documentclass[a4paper, 12pt]{report}
% --- preamble.tex ---
% 文件的導言區設定集合

% 版面設定
\usepackage[a4paper,left=2cm,right=2cm,top=2cm,bottom=2cm]{geometry}
% \usepackage[a4paper, margin=2.5cm]{geometry}
% \geometry{a4paper,left=2cm,right=2cm,top=2cm,bottom=2cm}

% 常用套件1
\usepackage{graphicx}
\graphicspath{{images/}{figures/}} % 設定圖片路徑
\usepackage[hidelinks]{hyperref} % 去除超連結顏色與框線
\usepackage{xcolor}
\usepackage{color} % 顏色套件
\usepackage{cascadia-code} % Cascadia Code 字型套件
\usepackage{float} % 用於強制圖片位置的套件

% 設定字型
% 文件語言與字型
\usepackage{fontspec}     %allow setting fonts
\usepackage{xeCJK}        % 中文套件
\usepackage{lstfiracode} % https://ctan.org/pkg/lstfiracode
\setmonofont{Cascadia Code}[Contextuals=Alternate,Ligatures=TeX,Scale=0.85] % 開啟連字效果,字型縮放至 90%
\setmainfont{Times New Roman}[Contextuals=NoAlternate,Ligatures=TeX,Scale=1] % 關閉連字效果,字型縮放至 90%
\setCJKmonofont{標楷體}[AutoFakeBold=3,ItalicFont={標楷體},AutoFakeSlant=.2]  % 設定中文字型,模擬粗體字與斜體字效果,數值越大,模擬的粗度越高,但效果可能越差 % 設定中文等寬字型
\setCJKmainfont{標楷體}[AutoFakeBold=3,ItalicFont={標楷體},AutoFakeSlant=.2]  % 設定中文字型,模擬粗體字與斜體字效果,數值越大,模擬的粗度越高,但效果可能越差
\XeTeXlinebreaklocale "zh" % 
\XeTeXlinebreakskip = 0pt plus 1pt
\hfuzz=10pt % 避免 overfull hbox 警告

% 程式碼排版設定
\usepackage{listings}
% 定義自訂的程式碼風格,取名為CodeStyle1
\lstdefinestyle{CodeStyle1}{
    upquote=true, % 顯示正確的單引號
    language=Python,  % C++, Java, Python, etc.
    style=FiraCodeStyle,
    basicstyle=\linespread{0.9}\ttfamily, % 設定為 90% 的行高, 等寬字型
    backgroundcolor=\color[gray]{0.96},
    showstringspaces=false,
    breaklines=true,    % 自動換行
    numberstyle=\footnotesize\color[gray]{0.5},
    numbers=left,
    morekeywords={const,arrow}, % 可添加更多關鍵字
    keywordstyle=\bfseries\color{blue},
    frame=single
}
% 全域程式碼排版設定
\lstset{
    language=Python,
    style=CodeStyle1,
}

% \renewcommand{\abstractname}{摘要}
% \renewcommand{\refname}{參考文獻}
\renewcommand{\contentsname}{\centering 目錄} % 修改目錄標題
\renewcommand{\listfigurename}{\centering 圖目錄} % 修改圖目錄標題
\renewcommand{\listtablename}{\centering 表目錄} % 修改表目錄標題
\renewcommand{\lstlistlistingname}{\centering 程式碼目錄} % 修改程式碼目錄標題
\renewcommand{\figurename}{圖} % 修改圖片標籤
\renewcommand{\tablename}{表} % 修改表格標籤
\renewcommand{\lstlistingname}{程式碼}  % 修改程式碼標籤

% 常用套件2
\usepackage{setspace} % 行距調整套件,可輕鬆設定單倍、1.5倍或雙倍行距,並能區域性或全域地應用
\setstretch{2.0} % 全域設定行距為 1.5 倍
% \singlespacing     % 單倍行距(預設狀態)
% \onehalfspacing    % 一點五倍行距
% \doublespacing     % 雙倍行距

% 章節樣式
\usepackage{titlesec} % 用於自定義章節標題格式
% \titleformat{\chapter}[hang]{\LARGE\bfseries}{第\ \thechapter\ 章}{1em}{} % 自定義章節標題格式
\titleformat{\chapter}[hang]{\LARGE\bfseries}{第\,\thechapter\,章}{1em}{} % \, 產生一個小空格(1/3em)

% 設定首段也縮排與行距
\usepackage{indentfirst}     % 讓首段也縮排(預設首段不縮)
\setlength{\parindent}{2em}  % 調整縮排距離,2em 約兩個中文字寬
\onehalfspacing


\title{專業排版與設計~期中考題 \\[2cm] LINE Messaging API 教學文件製作}
\author{座號: \makebox[4cm]{\begin{tabular}{c}\small 017\\[-3.0ex]\rule{4cm}{0.4pt}\end{tabular}} \\[1cm] 
班級: \makebox[4cm]{\begin{tabular}{c}\small 資管4A\\[-3.0ex]\rule{4cm}{0.4pt}\end{tabular}} \\[1cm] 
學號: \makebox[4cm]{\begin{tabular}{c}\small 411631277\\[-3.0ex]\rule{4cm}{0.4pt}\end{tabular}} \\[1cm] 
姓名: \makebox[4cm]{\begin{tabular}{c}\small 張誠恩\\[-3.0ex]\rule{4cm}{0.4pt}\end{tabular}} \\[1cm]}
\date{\today}

\begin{document} % 開始文件

\maketitle % 產生標題頁
\tableofcontents % 產生章節目錄
\listoffigures % 產生圖目錄

\begingroup % 開始一個新的群組,避免影響後續內容, 取消換頁
\let\clearpage\relax   % 重新定義 \clearpage 為不換頁
 %\listoftables % 產生表目錄
\lstlistoflistings % 產生程式碼目錄
\endgroup % 結束群組


\chapter{LINE Messaging API 簡介與應用場景}

\section{LINE Messaging API 概述}

LINE Messaging API 是由 LINE 公司提供的雲端服務介面,旨在讓開發者能以程式方式操作 LINE 聊天平台的互動功能。透過此 API,開發者可以建立能夠主動推播訊息、回覆使用者輸入、甚至與後端資料庫互動的自動化聊天系統。

Messaging API 採用 RESTful 架構,所有互動皆以 HTTPS 為基礎的 POST/GET 請求完成。開發者透過 LINE Developers 平台建立一個「Channel」,並使用「Channel Access Token」來進行身分驗證與授權。LINE 平台再透過 Webhook 機制,將使用者事件(例如訊息、加入好友、按鈕互動等)傳送至開發者設定的伺服器。

此技術的誕生使得企業不再需要自行開發聊天系統,而能以 LINE 為前端通道,將客戶溝通、行銷與服務集中於單一平台。這大幅降低開發成本,同時擴大應用覆蓋範圍。

%--分隔線--

\section{常見應用場景與發展趨勢}

Messaging API 的應用非常廣泛,以下列出幾個具代表性的實際場景:

\begin{enumerate}
    \item 智慧客服機器人:
    許多企業利用 LINE Bot 回覆常見問題,整合 FAQ、訂單查詢、報修狀態查詢等功能,減少人力成本。
    \item 行銷與推播通知:
    透過 Broadcast Message 或 Multicast API,企業可向特定用戶群發送促銷訊息,並根據互動紀錄分析點擊率。
    \item 服務整合平台:
    結合 Google Sheets、Firebase 或 MySQL,可實現會員查詢、預約系統、積分兌換等功能。
    \item 教育與社群應用
    學校可利用 LINE Bot 提供課程公告、作業提醒,社群團體亦能透過 Bot 管理活動報名。
\end{enumerate}
展望未來,LINE Messaging API 將持續朝「智慧化」、「個人化」與「跨平台整合」方向發展。AI 對話與使用者行為分析將成為關鍵技術,使 Bot 不僅能回覆問題,更能主動理解與預測使用者需求。

%--分隔線--
\chapter{LINE Developers 帳號註冊申請與 Channel 建立流程}

\section{LINE Developers 帳號建立}
要使用 Messaging API,第一步是註冊 LINE Developers 帳號。開發者可前往 https://developers.line.biz/
網站,以現有的 LINE 帳號登入。進入後可建立一個或多個「Provider」──代表開發者或公司身分。

建立 Provider 後,平台會要求設定名稱、聯絡信箱與用途描述。Provider 相當於開發者群組,而每個 Provider 底下可以擁有多個 Channel,用以區分不同應用程式或專案。

在帳號註冊完成後,開發者即可在主控台中看到專案清單與狀態,並能管理權限、API 使用紀錄等資訊。

%--分隔線--

\section{建立 Messaging API Channel 流程}

\begin{enumerate}
    \item 點擊「Create a new channel」
    \item 選擇類型:Messaging API
    \item 填寫 Channel 名稱、描述與類別
    \item 勾選同意使用條款
    \item 提交後即可在控制台看到新建的 Channel
    \end{enumerate}

    建立完成後,平台會自動生成:
     \begin{itemize}
        \item Channel ID
        \item Channel Secret
        \item Channel Access Token(可重新產生)
    \end{itemize}
    
    \begin{figure}[H] 
    \centering
    \includegraphics[scale=1]{line1.png}
    \caption{Create a new channel圖示}
\end{figure}

\vspace{1\baselineskip}

\begin{figure}[H]  
    \centering
    \includegraphics[scale=0.5]{line2.png}
    \caption{創建帳號}
\end{figure}

\vspace{1\baselineskip}

\begin{figure}[H]  
    \centering
    \includegraphics[scale=0.5]{line3.png}
    \caption{填寫 Channel 名稱、描述與類別}
\end{figure}

    這些資訊是後續整合 API 的關鍵。
此外,Channel 頁面也提供多項設定,如 Webhook URL、功能權限、回覆模式(自動或手動)等。

在實務開發中,建議將 Channel 區分為「開發測試用」與「正式營運用」兩組,以便版本控制與權限管理。

%--分隔線--

\chapter{Channel Secret 及 Access Token 取得與設定}

\section{建立 Messaging API Channel 流程}

Channel Secret 是由 LINE 平台自動產生的一串字串,用來驗證 Webhook 的來源真偽。
每當 LINE 伺服器發送事件通知至你的伺服器時,HTTP Header 內會附帶一個簽章(Signature)。
伺服器端應以 Channel Secret 進行 HMAC-SHA256 驗證,以確認請求未被偽造。

若 Secret 外洩,惡意者可能偽造事件請求並誘使伺服器誤判,因此務必妥善保管。常見作法是使用環境變數儲存,或以 .env 文件方式讀取。

%--分隔線--

\section{Access Token 的產生與管理}

Channel Access Token 是讓伺服器代表 LINE Bot 呼叫 API 的授權金鑰。
在 Messaging API 頁面中,開發者可選擇「Issue」產生新的 Token。建議選用長期有效(Long-lived)的 Token,以避免頻繁更新。

\vspace{1\baselineskip}

範例設定方式如下(Python Flask 環境):

\begin{lstlisting}[caption={Python Flask 環境 範例}, label={lst:app.py}]
export LINE_CHANNEL_SECRET="abcdef1234567890"
export LINE_CHANNEL_ACCESS_TOKEN="E9qU3Y3xyz..."
\end{lstlisting}

此 Token 在應用程式中用於初始化 LineBotApi 物件。
若需要強化安全性,可定期重新產生 Token 並更新伺服器設定。

\vspace{1\baselineskip}

此外,也建議啟用伺服器防火牆與 HTTPS 憑證,以確保通訊安全。

%--分隔線--

\chapter{Webhook 的設定與測試方式(可使用 ngrok 或伺服器)}

\section{Webhook 機制說明}
Webhook 是 Messaging API 的核心之一。當使用者在 LINE 對話中輸入文字或觸發事件時,LINE 伺服器會立即將該事件以 JSON 格式發送至你設定的 Webhook URL。

\vspace{1\baselineskip}

Webhook 的優點在於:
     \begin{itemize}
        \item 即時性高:可立即回應使用者操作。
        \item 節省輪詢資源:伺服器不需主動查詢,僅在事件發生時處理。
        \item 易於整合:可與資料庫、AI 模組或外部 API 結合。
    \end{itemize}

Webhook 的資料結構包括事件類型(如 message、follow、unfollow、join 等)以及使用者 ID、訊息內容、時間戳記等欄位。

%--分隔線--

\section{Webhook 機制說明}
若開發環境在本地端,無法被外部連線,此時可利用 ngrok 工具。

\vspace{1\baselineskip}

安裝後,執行以下命令:
\begin{lstlisting}[caption={ngrok指令1}, label={lst:app.py}]
ngrok http 5000
\end{lstlisting}

ngrok 會建立一條安全通道,並產生一個 HTTPS 公開網址,例如:
\begin{lstlisting}[caption={ngrok指令2}, label={lst:app.py}]
https://abcd1234.ngrok.io
\end{lstlisting}

將此網址填入 LINE Developers 控制台的「Webhook URL」欄位,然後點擊「Verify」按鈕。若顯示成功,即表示本地伺服器能正常接收事件。

\vspace{1\baselineskip}

在測試過程中,可於 Flask 伺服器輸出 print() 訊息以觀察事件內容,協助除錯。

%--分隔線--

\chapter{範例程式說明(接收及回覆 LINE 訊息)}

\section{程式架構介紹}

本範例使用 Python Flask 框架,搭配官方提供的"line-bot-sdk"套件。整體流程如下:
       \begin{enumerate}
        \item 使用者在 LINE 中傳送訊息給 Bot。
        \item LINE 伺服器透過 Webhook 將事件傳送至 Flask /callback 路由。
        \item Flask 接收到事件後交由 WebhookHandler 驗證簽章。
        \item 若為文字訊息事件,則以相同文字回覆。
    \end{enumerate}

此結構簡單明瞭,適合初學者快速上手。

%--分隔線--

\section{程式碼與執行步驟}

安裝必要套件:

\begin{lstlisting}[caption={安裝必要條件}, label={lst:app.py}]
pip install flask line-bot-sdk
\end{lstlisting}

範例主程式如下:

\begin{lstlisting}[caption={範例主程式}, label={lst:app.py}]
from flask import Flask, request, abort
from linebot import LineBotApi, WebhookHandler
from linebot.exceptions import InvalidSignatureError
from linebot.models import MessageEvent, TextMessage, TextSendMessage

app = Flask(__name__)

line_bot_api = LineBotApi('YOUR_CHANNEL_ACCESS_TOKEN')
handler = WebhookHandler('YOUR_CHANNEL_SECRET')

@app.route("/callback", methods=['POST'])
def callback():
    signature = request.headers['X-Line-Signature']
    body = request.get_data(as_text=True)
    try:
        handler.handle(body, signature)
    except InvalidSignatureError:
        abort(400)
    return 'OK'

@handler.add(MessageEvent, message=TextMessage)
def handle_message(event):
    msg = event.message.text
    reply = f"你說了: {msg}"
    line_bot_api.reply_message(
        event.reply_token,
        TextSendMessage(text=reply)
    )

if __name__ == "__main__":
    app.run(port=5000)

\end{lstlisting}

執行後搭配 ngrok 測試,即可於手機上體驗 Bot 的回覆行為。

\vspace{1\baselineskip}

若要擴充,可加入圖片訊息(ImageMessage)、按鈕模板或 Flex Message。

%--分隔線--

\chapter{結論與未來發展}

\section{開發成果總結}

本文件完整說明了 LINE Messaging API 的基礎原理、開發流程、安全設定與測試方式,並提供實際程式範例。
透過此流程,開發者能從無到有建立一個能與使用者互動的 LINE Bot,並掌握從 Channel 設定到 Webhook 驗證的關鍵步驟。

\vspace{1\baselineskip}

本範例雖屬基礎型聊天機器人,但其架構具延展性,可應用於各類智慧服務,例如即時客服、智慧助理、行銷推播與 IoT 裝置整合等。

\section{未來應用方向與挑戰}
未來的 LINE Bot 開發趨勢將聚焦於以下方向:
     \begin{enumerate}
        \item 人工智慧整合:結合 ChatGPT、Dialogflow 等 NLP 模型,提供更自然的對話體驗。
        \item 多媒體互動:支援圖片、影片、地圖、音訊回覆,豐富使用者體驗。
        \item 資料分析與行銷決策:藉由使用者互動紀錄進行行為分析,優化產品策略。
        \item 跨平台串接:將 LINE 與網站、App、CRM 系統互通,形成完整的服務生態。
    \end{enumerate}
    
不過,開發者也必須面對如使用者隱私保護、API 權限管理與伺服器安全等挑戰。
唯有兼顧便利與安全,才能打造穩定、可信賴的智慧對話應用。

\end{document}

