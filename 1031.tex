\documentclass[a4paper, 12pt]{report}
% --- preamble.tex ---
% 文件的導言區設定集合

% 版面設定
\usepackage[a4paper,left=2cm,right=2cm,top=2cm,bottom=2cm]{geometry}
% \usepackage[a4paper, margin=2.5cm]{geometry}
% \geometry{a4paper,left=2cm,right=2cm,top=2cm,bottom=2cm}

% 常用套件1
\usepackage{graphicx}
\graphicspath{{images/}{figures/}} % 設定圖片路徑
\usepackage[hidelinks]{hyperref} % 去除超連結顏色與框線
\usepackage{xcolor}
\usepackage{color} % 顏色套件
\usepackage{cascadia-code} % Cascadia Code 字型套件
\usepackage{float} % 用於強制圖片位置的套件

% 設定字型
% 文件語言與字型
\usepackage{fontspec}     %allow setting fonts
\usepackage{xeCJK}        % 中文套件
\usepackage{lstfiracode} % https://ctan.org/pkg/lstfiracode
\setmonofont{Cascadia Code}[Contextuals=Alternate,Ligatures=TeX,Scale=0.85] % 開啟連字效果,字型縮放至 90%
\setmainfont{Times New Roman}[Contextuals=NoAlternate,Ligatures=TeX,Scale=1] % 關閉連字效果,字型縮放至 90%
\setCJKmonofont{標楷體}[AutoFakeBold=3,ItalicFont={標楷體},AutoFakeSlant=.2]  % 設定中文字型,模擬粗體字與斜體字效果,數值越大,模擬的粗度越高,但效果可能越差 % 設定中文等寬字型
\setCJKmainfont{標楷體}[AutoFakeBold=3,ItalicFont={標楷體},AutoFakeSlant=.2]  % 設定中文字型,模擬粗體字與斜體字效果,數值越大,模擬的粗度越高,但效果可能越差
\XeTeXlinebreaklocale "zh" % 
\XeTeXlinebreakskip = 0pt plus 1pt
\hfuzz=10pt % 避免 overfull hbox 警告

% 程式碼排版設定
\usepackage{listings}
% 定義自訂的程式碼風格,取名為CodeStyle1
\lstdefinestyle{CodeStyle1}{
    upquote=true, % 顯示正確的單引號
    language=Python,  % C++, Java, Python, etc.
    style=FiraCodeStyle,
    basicstyle=\linespread{0.9}\ttfamily, % 設定為 90% 的行高, 等寬字型
    backgroundcolor=\color[gray]{0.96},
    showstringspaces=false,
    breaklines=true,    % 自動換行
    numberstyle=\footnotesize\color[gray]{0.5},
    numbers=left,
    morekeywords={const,arrow}, % 可添加更多關鍵字
    keywordstyle=\bfseries\color{blue},
    frame=single
}
% 全域程式碼排版設定
\lstset{
    language=Python,
    style=CodeStyle1,
}

% \renewcommand{\abstractname}{摘要}
% \renewcommand{\refname}{參考文獻}
\renewcommand{\contentsname}{\centering 目錄} % 修改目錄標題
\renewcommand{\listfigurename}{\centering 圖目錄} % 修改圖目錄標題
\renewcommand{\listtablename}{\centering 表目錄} % 修改表目錄標題
\renewcommand{\lstlistlistingname}{\centering 程式碼目錄} % 修改程式碼目錄標題
\renewcommand{\figurename}{圖} % 修改圖片標籤
\renewcommand{\tablename}{表} % 修改表格標籤
\renewcommand{\lstlistingname}{程式碼}  % 修改程式碼標籤

% 常用套件2
\usepackage{setspace} % 行距調整套件,可輕鬆設定單倍、1.5倍或雙倍行距,並能區域性或全域地應用
\setstretch{2.0} % 全域設定行距為 1.5 倍
% \singlespacing     % 單倍行距(預設狀態)
% \onehalfspacing    % 一點五倍行距
% \doublespacing     % 雙倍行距

% 章節樣式
\usepackage{titlesec} % 用於自定義章節標題格式
% \titleformat{\chapter}[hang]{\LARGE\bfseries}{第\ \thechapter\ 章}{1em}{} % 自定義章節標題格式
\titleformat{\chapter}[hang]{\LARGE\bfseries}{第\,\thechapter\,章}{1em}{} % \, 產生一個小空格(1/3em)

% 設定首段也縮排與行距
\usepackage{indentfirst}     % 讓首段也縮排(預設首段不縮)
\setlength{\parindent}{2em}  % 調整縮排距離,2em 約兩個中文字寬
\onehalfspacing


\title{專業排版與設計~期中考題 \\[2cm] LINE Messaging API 教學文件製作}
\author{座號: \makebox[4cm]{\begin{tabular}{c}\small 017\\[-3.0ex]\rule{4cm}{0.4pt}\end{tabular}} \\[1cm] 
班級: \makebox[4cm]{\begin{tabular}{c}\small 資管4A\\[-3.0ex]\rule{4cm}{0.4pt}\end{tabular}} \\[1cm] 
學號: \makebox[4cm]{\begin{tabular}{c}\small 411631277\\[-3.0ex]\rule{4cm}{0.4pt}\end{tabular}} \\[1cm] 
姓名: \makebox[4cm]{\begin{tabular}{c}\small 張誠恩\\[-3.0ex]\rule{4cm}{0.4pt}\end{tabular}} \\[1cm]}
\date{\today}

\begin{document} % 開始文件

\maketitle % 產生標題頁
\tableofcontents % 產生章節目錄
\listoffigures % 產生圖目錄

\begingroup % 開始一個新的群組,避免影響後續內容, 取消換頁
\let\clearpage\relax   % 重新定義 \clearpage 為不換頁
\listoftables % 產生表目錄
\lstlistoflistings % 產生程式碼目錄
\endgroup % 結束群組


\chapter{LINE Messaging API 簡介與應用場景}

\section{LINE Messaging API 概述}

LINE Messaging API 是由 LINE 公司提供的雲端服務介面,旨在讓開發者能以程式方式操作 LINE 聊天平台的互動功能。透過此 API,開發者可以建立能夠主動推播訊息、回覆使用者輸入、甚至與後端資料庫互動的自動化聊天系統。

Messaging API 採用 RESTful 架構,所有互動皆以 HTTPS 為基礎的 POST/GET 請求完成。開發者透過 LINE Developers 平台建立一個「Channel」,並使用「Channel Access Token」來進行身分驗證與授權。LINE 平台再透過 Webhook 機制,將使用者事件(例如訊息、加入好友、按鈕互動等)傳送至開發者設定的伺服器。

此技術的誕生使得企業不再需要自行開發聊天系統,而能以 LINE 為前端通道,將客戶溝通、行銷與服務集中於單一平台。這大幅降低開發成本,同時擴大應用覆蓋範圍。

%--分隔線--

\section{常見應用場景與發展趨勢}

Messaging API 的應用非常廣泛,以下列出幾個具代表性的實際場景:

\begin{enumerate}
    \item 智慧客服機器人:
    許多企業利用 LINE Bot 回覆常見問題,整合 FAQ、訂單查詢、報修狀態查詢等功能,減少人力成本。
    \item 行銷與推播通知:
    透過 Broadcast Message 或 Multicast API,企業可向特定用戶群發送促銷訊息,並根據互動紀錄分析點擊率。
    \item 服務整合平台:
    結合 Google Sheets、Firebase 或 MySQL,可實現會員查詢、預約系統、積分兌換等功能。
    \item 教育與社群應用
    學校可利用 LINE Bot 提供課程公告、作業提醒,社群團體亦能透過 Bot 管理活動報名。
\end{enumerate}
展望未來,LINE Messaging API 將持續朝「智慧化」、「個人化」與「跨平台整合」方向發展。AI 對話與使用者行為分析將成為關鍵技術,使 Bot 不僅能回覆問題,更能主動理解與預測使用者需求。


\chapter{LINE Developers 帳號註冊申請與 Channel 建立流程}

\section{LINE Developers 帳號建立}
要使用 Messaging API,第一步是註冊 LINE Developers 帳號。開發者可前往 https://developers.line.biz/
網站,以現有的 LINE 帳號登入。進入後可建立一個或多個「Provider」──代表開發者或公司身分。

建立 Provider 後,平台會要求設定名稱、聯絡信箱與用途描述。Provider 相當於開發者群組,而每個 Provider 底下可以擁有多個 Channel,用以區分不同應用程式或專案。

在帳號註冊完成後,開發者即可在主控台中看到專案清單與狀態,並能管理權限、API 使用紀錄等資訊。

% 圖片 -----------------
\begin{figure}[ht]
    \centering
    \includegraphics[scale=0.2]{photo1.png}
    \caption{圖片範例展示}
\end{figure}


\chapter{Channel Secret 及 Access Token 取得與設定}

cccccc


\chapter{Webhook 的設定與測試方式(可使用 ngrok 或伺服器)}

dddddd

\chapter{範例程式說明(接收及回覆 LINE 訊息)}
\begin{lstlisting}[caption={Python app.py 範例}, label={lst:app.py}]
from flask import Flask, request, abort
import os
from linebot.v3 import WebhookHandler
from linebot.v3.messaging import Configuration, ApiClient, MessagingApi, ReplyMessageRequest, TextMessage
from linebot.v3.exceptions import InvalidSignatureError
from linebot.v3.webhooks import MessageEvent, TextMessageContent
from dotenv import load_dotenv

# 載入環境變數
load_dotenv()

app = Flask(__name__)

# 讀取 LINE Credentials 從環境變數
CHANNEL_SECRET = os.getenv("LINE_CHANNEL_SECRET")
CHANNEL_ACCESS_TOKEN = os.getenv("LINE_CHANNEL_ACCESS_TOKEN")
\end{lstlisting}

\end{document}

