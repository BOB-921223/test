% --- preamble.tex ---
% 文件的導言區設定集合

% 版面設定
\usepackage[a4paper,left=2cm,right=2cm,top=2cm,bottom=2cm]{geometry}
% \usepackage[a4paper, margin=2.5cm]{geometry}
% \geometry{a4paper,left=2cm,right=2cm,top=2cm,bottom=2cm}

% 常用套件1
\usepackage{graphicx}
\graphicspath{{images/}{figures/}} % 設定圖片路徑
\usepackage[hidelinks]{hyperref} % 去除超連結顏色與框線
\usepackage{xcolor}
\usepackage{color} % 顏色套件
\usepackage{cascadia-code} % Cascadia Code 字型套件
\usepackage{float} % 用於強制圖片位置的套件

% 設定字型
% 文件語言與字型
\usepackage{fontspec}     %allow setting fonts
\usepackage{xeCJK}        % 中文套件
\usepackage{lstfiracode} % https://ctan.org/pkg/lstfiracode
\setmonofont{Cascadia Code}[Contextuals=Alternate,Ligatures=TeX,Scale=0.85] % 開啟連字效果,字型縮放至 90%
\setmainfont{Times New Roman}[Contextuals=NoAlternate,Ligatures=TeX,Scale=1] % 關閉連字效果,字型縮放至 90%
\setCJKmonofont{標楷體}[AutoFakeBold=3,ItalicFont={標楷體},AutoFakeSlant=.2]  % 設定中文字型,模擬粗體字與斜體字效果,數值越大,模擬的粗度越高,但效果可能越差 % 設定中文等寬字型
\setCJKmainfont{標楷體}[AutoFakeBold=3,ItalicFont={標楷體},AutoFakeSlant=.2]  % 設定中文字型,模擬粗體字與斜體字效果,數值越大,模擬的粗度越高,但效果可能越差
\XeTeXlinebreaklocale "zh" % 
\XeTeXlinebreakskip = 0pt plus 1pt
\hfuzz=10pt % 避免 overfull hbox 警告

% 程式碼排版設定
\usepackage{listings}
% 定義自訂的程式碼風格,取名為CodeStyle1
\lstdefinestyle{CodeStyle1}{
    upquote=true, % 顯示正確的單引號
    language=Python,  % C++, Java, Python, etc.
    style=FiraCodeStyle,
    basicstyle=\linespread{0.9}\ttfamily, % 設定為 90% 的行高, 等寬字型
    backgroundcolor=\color[gray]{0.96},
    showstringspaces=false,
    breaklines=true,    % 自動換行
    numberstyle=\footnotesize\color[gray]{0.5},
    numbers=left,
    morekeywords={const,arrow}, % 可添加更多關鍵字
    keywordstyle=\bfseries\color{blue},
    frame=single
}
% 全域程式碼排版設定
\lstset{
    language=Python,
    style=CodeStyle1,
}

% \renewcommand{\abstractname}{摘要}
% \renewcommand{\refname}{參考文獻}
\renewcommand{\contentsname}{\centering 目錄} % 修改目錄標題
\renewcommand{\listfigurename}{\centering 圖目錄} % 修改圖目錄標題
\renewcommand{\listtablename}{\centering 表目錄} % 修改表目錄標題
\renewcommand{\lstlistlistingname}{\centering 程式碼目錄} % 修改程式碼目錄標題
\renewcommand{\figurename}{圖} % 修改圖片標籤
\renewcommand{\tablename}{表} % 修改表格標籤
\renewcommand{\lstlistingname}{程式碼}  % 修改程式碼標籤

% 常用套件2
\usepackage{setspace} % 行距調整套件,可輕鬆設定單倍、1.5倍或雙倍行距,並能區域性或全域地應用
\setstretch{2.0} % 全域設定行距為 1.5 倍
% \singlespacing     % 單倍行距(預設狀態)
% \onehalfspacing    % 一點五倍行距
% \doublespacing     % 雙倍行距

% 章節樣式
\usepackage{titlesec} % 用於自定義章節標題格式
% \titleformat{\chapter}[hang]{\LARGE\bfseries}{第\ \thechapter\ 章}{1em}{} % 自定義章節標題格式
\titleformat{\chapter}[hang]{\LARGE\bfseries}{第\,\thechapter\,章}{1em}{} % \, 產生一個小空格(1/3em)

% 設定首段也縮排與行距
\usepackage{indentfirst}     % 讓首段也縮排(預設首段不縮)
\setlength{\parindent}{2em}  % 調整縮排距離,2em 約兩個中文字寬
\onehalfspacing
